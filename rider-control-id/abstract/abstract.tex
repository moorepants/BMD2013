\documentclass{article}

\usepackage{fullpage}
\usepackage{graphicx}

\title{Identification of manual control employed during bicycling}
\author{Jason K. Moore, Mont Hubbard, and Ronald Hess}
\date{\today}

\begin{document}

\maketitle

When balancing and directing a bicycle the human rider senses his or her motion
and the environment an then actuates his or her body to cause the bicycle to
travel in the desired direction. This requires both  stabilization, as the
bicycle-rider system is an unstable system, and path following. The most
effective control input for controlling a bicycle under typical operation is by
applying forces to cause the front frame to rotate about the steering axis but
riders are also capable of using body language to enact control of a bicycle.
It is possible to predict the control actions of the rider using manual control
theory and there has been much work to validate this with vehicles such as
aircraft and automobiles. But there have been just a few attempts to identify
the rider's control actions while controlling bicycles or motorcycles.

We have collected a large set of time history data from an instrumented bicycle
which includes the most important kinematic and kinetic variables to describe
the bicycle-rider motion from three different riders on the same bicycle for a
variety of speeds. Furthermore, the instrumented bicycle was designed so that
the riders were not able to move their legs or torso relative to the rear frame
of the bicycle, to ensure that the assumption of rider rigidity of the Whipple
bicycle model was as close to valid as possible and to enforce a single control
input from the rider. In the experiments we perturb the bicycle-rider system
with an externally applied lateral force and measured the rider's response.

With the single-input multi-output data set in mind we formulate an 8th order
grey box state space model in the directly parameterized innovations form. The
model is made up of the plant and the controller. Due to the poor predictive
ability of the Whipple bicycle model we made use of a bicycle-rider system
model identified are larger superset of the data used here. We then combine
this model with a 2nd order model of the rider's neuromuscular system to form
the plant. The controller structure is taken from \cite{Hess2012} which uses
five gains in sequential feedback loops each simulating realistic sensory cues
used by the rider.

We then identify the unknown controller gains for each of the runs using the
prediction error method, giving system models that predict the state
trajectories with an average of $62 \pm 12$ percent of the variance accounted
for over all the identified runs \ref{fig}. The resulting models are then
analyzed and shown to hold well to the theory presented in \cite{Hess2012}.

\begin{figure}
  \centering
  \includegraphics[width=5.0in]{rider-id-treadmill-run.pdf}
  \caption{Simulation of an identified model derived from the inputs and
  outputs (SIMO) of one of Charlie's treadmill runs \#288 (4.23 m/s) validated
  against the data from run \#289 (4.22 m/s). The black line is the processed
  and filtered (low pass 15 Hz) measured data, the blue line is the simulation
  from the identified SIMO model and the green line is the identified SISO
  model.}
  \label{fig}
\end{figure}

We show that a simple rider controller can be identified from the collected
data given that the plant model of the bicycle/rider system is properly chosen,
in our case identified separately from the same data. In addition, these basic
conclusion arise:

\begin{itemize}
  \item The fundamental, remnant-free, control response of the rider under
    lateral perturbations can be described reasonably well by the simple five
    gain sequential loop closure and an eighth order closed loop system. No
    time delays are needed and the continuous formulation is adequate for good
    prediction.
  \item The identified gains seem to exhibit linear trends with respect to
    speed as predicted by theory and the identified neuromuscular frequency
    seems to be constant with a median around the theoretical prediction of 30
    rad/s.
  \item The identified parameters show resemblance to the patterns in the
    theoretical loop closure techniques, especially in that the rider selects
    their gains such that the closed roll rate loop exhibits a 10 dB peak
    around 10-11 rad/s and that the riders cross over the outer three loops in
    the predicted order.
  \item The crossover frequencies of the three outer loops are relatively
    constant with respect to speed and point to a speed independence of system
    response bandwidth selection among riders in this task.
\end{itemize}

This paper is based on work supported by the National Science Foundation under
Grant No 0928339.

\bibliographystyle{plain}
\bibliography{bicycle}

\end{document}
