\documentclass{article}

\title{Identification of the open loop dynamics of a bicycle-rider system
under manual control}
\author{Jason K. Moore and Mont Hubbard}
\date{\today}

\begin{document}

\maketitle

\section{Introduction}

- state of bicycle/rider models
- short review on identification of the system

Traditionally, dynamicists develop models of the bicycle-rider system from
first principles, i.e. Newton's laws and various other atomic fundamental
models of nature, e.g. friction, springs, etc. The first principle approach has
guided much of modern engineering throughout history but today's dynamical
experiments are capable of delivering a staggering amount of both kinematic and
kinetic data from complex dynamical systems. This large amount lends itself to
data driven modeling approaches that can potentially provide more predictive
models than the traditional buidling blocks of dynamical models. These data
driven models can also give insight into the defencies of first principles
models. Here we explore the bicycle-rider system and discuss both traditional
models and data drive models.

Bicycles and the bicycle-rider system have been modeled by a variety of models,
both simple \cite{Karnopp or Timoshenko} and complex \cite{Sharp}. Many of today's
studies rely on the benchmarked Whipple model, with or without the additional
of tire models, for analtyical studies of the system and simulation
comparisons. The Whipple bicycle model is regarded as a highly predictive model
of the bicycle-rider system and is constructed from first principles, yet very
little experimental data proves that the Whipple model is in fact a robust
model for the open loop dynamics of the bicycle-rider system. There are also a
variety of popular motorcycle models (Sharp, Cosalter) that may be a good
predictor of the system dynamics, but no one has attempted to verify that for a
bicycle-rider system. The author is only aware of two significant experimental
attempts at validating the model. The most cited, Kooijman, shows that the
Whipple model is predictive of a \bold{riderless} bicycle in a gymnasium and on
a treadmill for speeds in its stable speed range, 4-6 m/s. Cain shows that the
Whipple model reduced to a linear steady turn can predict the kinematics, but
is not so good at predicting the rider's input torque.

It is worth noting that Eaton, James, \ldots, worked to identify models of
motrocycles with varying degrees of results. Talk more about these.

Two remedy this, we have collected a large set of mostly time history data from
an instrumented bicycle which includes some of the most important kinematic and
kinetic variables describing the bicycle-rider motion from three different
riders on the same bicycle for a variety of maneuvers and speeds. These
experiments generated about 1.7 million time samples from each of about 30
sensors collected at 200 hertz (representing about 2.4 hours of real time).

\section{Experimental Design}

\subsection{Instrumented Bicycle}
We developed an instrumented bicycle that had a unique combination of sensors
for dynamic measurements that are known to be important indicators for the
bicycle-rider system motion. The bicycle's primary design criteria were as
follows:

\begin{itemize}
  \item It should be sized for our intended riders: average adult males.
  \item The rider's biomechanical movement, including pedaling, should be
    restricted as much as possible so that the Whipple model's rigid rider
    assumption is more probable.
  \item It should accurately measure the rider's applied steering torque.
  \item It should accurately measure the fundamental kinematics of the bicycle:
    three dimensional rates and orientations of the bicycle rear frame, front
    frame, and wheels.
  \item It should accurately measure an applied lateral disturbance force to
    the bicycle frame.
  \item Experiments could be performed on open road or on a treadmill.
\end{itemize}

With these criteria in mind we constructed a bicycle with an electric
propulsion system and rigid rider harness and an assortment of sensors. The
rear frame 3D angular rates and a 3D point acceleration were measured with a
VectorNav VN-100 interial measurement unit, the rear frame roll angle was
measured with a rotary potentiometer mounted to a small lightweight trailer,
the steer angle was measured with a rotary potentiometer, the axial torque in
the steer tube by a Futek Torque Sensor, the lateral perturbation force by a
load cell, the angular rate about the steer axis of the front frame by a rate
gyro, and the rear wheel angular rate by a dc generator.

% Labeled photo of the bicycle

\subsection{Experiments}
I only focus on the Balance and Track Straight Line maneuvers with and without
disturbances in the following analyses and they will be referred to as Heading
Tracking and Lateral Deviation Tracking in the text (as opposed to the labels
in the database).

Heading Tracking
    The rider was instructed to simply balance the bicycle and keep a
    relatively constant heading while focusing their vision at a point
    in the far distance.
Lateral Deviation Tracking
    The rider was instructed to focus on a straight line that was marked
    on the ground and to attempt to keep the front wheel on the line.

Both tasks were performed with and without the application of a manually
applied lateral perturbation force just below the seat. The forces were
applied randomly in direction and time.

\section{Data}

The experimental data was collected on seven different days. The first few days
were mostly trials to test the equipment, procedures and different maneuvers.
The data from the trial days is valid data and we ended up using it in our
analysis. The tires were pumped to 100 psi at the start of each day.

February 4 2011 Runs 103-109
   These were the first trials on the treadmill for preliminary testing. Only
   Jason rode. We performed lateral deviation tracking with disturbances. The
   bike fell over, broke and we had to cut it short.
February 28, 2011 Run 115-170
   These were the first trials in the pavilion. Jason was the only rider. We
   tried lane changes (115-139), lateral deviation tracking with disturbances
   (140-157), and a mixture of heading tracking and lateral deviation tracking
   with no disturbances (158-170). I noted that the slip clutch backlash seemed
   to be larger than the previous day with a guess of about 1 degree.
March 9, 2011 Runs 180-204
   This was the second go at the treadmill, still just testing things. Jason
   was the only rider. We did heading and lateral deviation tracking with
   disturbances and some lane changes. The lane changes were 0.25 m wide left
   and right maneuvers back and forth among two lines on the treadmill at 2 m
   long segments. Countdown markers to give an idea when the lane change
   started were necessary due to the rider's limited preview distance. We did
   the highest speed during any subsequent trials at 9 m/s. The 9 m/s runs
   generated a large amount of noise in the lateral force channel. The treadmill
   elevation was set at 0.1% upwards incline (because it was stuck).
August 30, 2011 Runs 235-291
   Jason and Luke rode and performed heading and lateral deviation tasks with
   and without perturbations at three speeds on the treadmill.
September 6, 2011 Runs 295-318
   Charlie performed heading and lateral deviation tasks with and without
   perturbations on the treadmill.
September 9, 2011 Runs 325-536
   Luke, Charlie and Jason performed heading and lateral deviation tracking
   tasks on the Pavilion floor with and without perturbations. Most of Luke and
   Charlie's runs were corrupt due to the time synchronization issues.
September 21, 2011 Runs 538-706
   Luke and Charlie repeated the runs from September 9th. And we added a couple
   of blind runs for each of them.

The meta data and raw time history data for each run and all sensor calibration
data were stored in individual Matlab mat files on the data acquisition
computer using the `BicycleDAQ <https://github.com/moorepants/BicycleDAQ>`_
software. The run files and calibration files are automatically numbered in
sequence with a five digit number; one sequence for runs and one for
calibrations. These mat files were then parsed and merged into a uniform,
organized, and complete single HDF5 database that could be accessed by a number
of programs and languages for fast data queries. I made use of `PyTables
<http://www.pytables.org>`_ for writing and reading from the database. The
software `BicycleDataProcessor
<http://github.com/moorepants/BicycleDataProcessor>`_ was designed as an
interface to the data in the database. In particular, it is able to load the
raw data from individual runs, process it, and present it for easy manipulation
and viewing.

The database is initially structured with three top-level tables and nodes
containing the time histories of the sensors for each run. The run table has a
row for each run and the columns store each piece of meta data, including the
corruption coding described below. The signal table has a row for each raw and
processed signal type and the classification information for each. The
calibration table has a row for each calibration which provides information
about the sensor and the data collected in the calibration.

We recorded a large set of meta data for each run to help with parsing during
analyses. We also video recorded all of the runs (minus a few video mishaps).
I coded each run based on the notes, data quality, and viewing the video for
potential or definite corrupted data with the following five codes.

Corrupt
   If the data is completely unusable due to time synchronization issues or
   others then this is set to true.
Warning
   Runs with a warning flag are questionable and potentially not usable.
Knee
   The rider's knees would sometimes de-clip from the frame during a
   perturbation. This potentially invalidates the rigid rider assumption. An
   array of 15 boolean values, one for each perturbation in the run, are stored
   for each run and each true value in the array represents an individual
   perturbation where a knee disengaged with the bicycle.
Handlebar
   On the treadmill the bicycle handlebars occasionally contacted the side
   railings. Each perturbation during the run in which this happened was
   recorded.
Trailer
   On the treadmill the roll trailer occasionally contacted the side of the
   treadmill. Each perturbation during the run which this happened was
   recorded.

We ultimately collected 600+ runs that were potentially usable for analysis.
:ref:`Figure 11.4<figDataBarPlots>` gives a breakdown of the runs by rider,
environment, maneuvers, and speed bins.
\section{Analyses}

- data processing
- state space form
- canonical form

\section{Discussion}


\section{Acknowledgements}

This paper is based on work supported by the National Science Foundation under
Grant No 0928339. Karl Astrom gave me the ideas for the canonical form.

\end{document}
