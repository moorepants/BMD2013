\documentclass{article}

\title{Identification of the open loop dynamics of a bicycle-rider system
under manual control}
\author{Jason K. Moore and Mont Hubbard}
\date{\today}

\begin{document}

\maketitle

\section{Introduction}

- state of bicycle/rider models
- short review on identification of the system

Traditionally, dynamicists develop models of the bicycle-rider system from
first principles, i.e. Newton's laws and various other atomic fundamental
models of nature, e.g. friction, springs, etc. The first principle approach has
guided much of modern engineering throughout history but today's dynamical
experiments are capable of delivering a staggering amount of both kinematic and
kinetic data. This large amount lends itself to data driven modeling
approaches that can potentially provide more predictive models than the
traditional buidling blocks of dynamical models. These data driven models can
also give insight into the defencies of first principles models.

Bicycles and the bicycle-rider system have been modeled by a variety of models
and today's studies typically rely on the benchmarked Whipple model, with or
without the additional of tire models, for analtyical studies of teh system.
The Whipple bicycle model is regarded as a highly predictive model of the
bicycle-rider system and is constructed from first principles, yet very little
experimental data proves that the Whipple model is in fact a robust model for
the open loop dynamics of the bicycle-rider system. There are also a variety of
popular motorcycle models (Sharp, Cosalter) that may be a good predictor of the
system dynamics, but no one has attempted to verify that for a bicycle-rider
system. The author is only aware of two significant experimental attempts at
validating the model. The most cited, Kooijman, shows that the Whipple model is
predictive of a \bold{riderless} bicycle in a gymnasium and on a treadmill for
speeds in its stable speed range, 4-6 m/s. Cain shows that the Whipple model
reduced to a linear steady turn can predict the kinematics, but is not so good
at predicting the rider's input torque.

It is worth noting that Eaton, James, \ldots, worked to identify models of
motrocycles with varying degrees of results.

Two remedy this, we have collected a large set of time history data from an
instrumented bicycle which includes the most important kinematic and kinetic
variables describing the bicycle-rider motion from three different riders on
the same bicycle for a variety of maneuvers and speeds. These experiments
generated about 1.7 million time samples from each of about 30 sensors
collected at 200 hertz (representing about 2.4 hours of real time).

\section{Experimental Design}

The instrumented bicycle was designed so that the riders were not able to move
their legs or torso relative to the rear frame of the bicycle, to ensure that
the assumption of rider rigidity of the Whipple bicycle model was as close to
valid as possible.

\section{Data}

\section{Analyses}

- data processing
- state space form
- canonical form

\section{Discussion}


\section{Acknowledgements}

This paper is based on work supported by the National Science Foundation under
Grant No 0928339.

Karl Astrom

\end{document}
